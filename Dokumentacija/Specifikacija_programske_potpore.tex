\chapter{Specifikacija programske potpore}
		
	\section{Funkcionalni zahtjevi}
			
			\noindent \textbf{Dionici:}
			
			\begin{packed_enum}
				
				\item Vlasnik (naručitelj)
				\item Stanovnik četvrti			
				\begin{packed_enum}
					\item  Obični stanovnik
					\item  Vijećnik
					\item  Moderator
				\end{packed_enum}
				\item Administrator	
				\item Razvojni tim
				
			\end{packed_enum}
			
			\noindent \textbf{Aktori i njihovi funkcionalni zahtjevi:}
			
			
			\begin{packed_enum}
			
				\item  \underbar{Neregistrirani korisnik (inicijator) može:}
				\begin{packed_enum}
					\item se registrirati u sustav stvaranjem novog korisničkog računa, za što su mu potrebni ime, prezime, adresa stanovanja, adresa e-pošte i lozinka
				\end{packed_enum}
				
				\item  \underbar{Neprijavljeni korisnik (inicijator) može:}
				\begin{packed_enum}
					\item se prijaviti u sustav,	za što su mu potrebni adresa e-pošte i lozinka
				\end{packed_enum}
				
				\item  \underbar{Stanovnik četvrti (inicijator) može:}
				\begin{packed_enum}
					\item čitati teme na "Forumu"
					\item otvarati teme na "Forumu"
					\item odgovarati u temama na "Forumu"
					\item uređivati svoje odgovore na "Forumu"
					\item ukloniti svoje odgovore na "Forumu"
					\item otvoriti temu na "Forumu" vezanu za objavu na "Vijeću četvrti", ako za tu objavu već nije otvorena tema
					\item čitati objave na "Vijeću četvrti"
					\item vidjeti najave događaja u cjelini "Događaji"
					\item predlagati najave budućih događaja u cjelini "Događaji", za što je potrebno navesti naziv, mjesto, vrijeme, trajanje i kratak opis
					\item poslati zahtjev administratorima za promjenu uloge (u "Vijećnika" ili "Moderatora")
					\item promijeniti osobne podatke	
					\item odjaviti se iz sustava
					\item obrisati svoj korisnički račun
				\end{packed_enum}
				
				\item  \underbar{Vijećnik (inicijator) može:}
				\begin{packed_enum}
					\item stvoriti objavu u cjelini "Vijeće četvrti"
					\item urediti objavu u cjelini "Vijeće četvrti"
					\item obrisati objavu u cjelini "Vijeće četvrti"	
				\end{packed_enum}
				
				\item  \underbar{Moderator (inicijator) može:}
				\begin{packed_enum}
					\item pregledati najave događanja koje predlažu stanovnici
					\item objaviti najave događanja u cjelini "Događaji"
					\item odbaciti prijedloge događanja
					\item urediti prijedloge događanja
					\item ukloniti odgovore u temama na "Forumu"
					\item ukloniti teme na Forumu
				\end{packed_enum}
				
				\item  \underbar{Administrator (inicijator) može:}
				\begin{packed_enum}
					\item vidjeti popis svih registriranih korisnika i njihove osobne podatke
					\item definirati četvrt i područje koje ta četvrt obuhvaća
					\item obrisati četvrti
					\item pristupiti svim zahtjevima za uloge vijećnika ili moderatora
					\item odbiti zahtjev za dodjelu uloge
					\item prihvatiti zahtjev za dodjelu uloge
					\item dodijeliti ulogu "Moderator" ili "Vijećnik" korisniku
					\item oduzeti ulogu "Moderator" ili "Vijećnik" korisniku
					\item privremeno blokirati korisnika
					\item pristupiti popisu blokiranih korisnika
					\item deblokirati korisnika
					\item trajno obrisati profil korisnika
					\item vidjeti popis svih četvrti u sustavu
					\item odabrati pojedinu četvrt s popisa četvrti i pristupiti svom korisničkom sadržaju te četvrti
				\end{packed_enum}
				
				\item  \underbar{Baza podataka (sudionik):}
				\begin{packed_enum}
					\item pohranjuje podatke o korisnicima i njihovim ovlastima
					\item pohranjuje podatke o četvrtima
					\begin{packed_enum}
						\item ulice koje im pripadaju
						\item teme na "Forumu"
						\item izvješća s "Vijeća četvrti"
					\end{packed_enum}
				\end{packed_enum}
				
			\end{packed_enum}
			
			\eject 
			
			
				
			\subsection{Obrasci uporabe}
				
				\textbf{\textit{dio 1. revizije}}
				
				\subsubsection{Opis obrazaca uporabe}
					\textit{Funkcionalne zahtjeve razraditi u obliku obrazaca uporabe. Svaki obrazac je potrebno razraditi prema donjem predlošku. Ukoliko u nekom koraku može doći do odstupanja, potrebno je to odstupanje opisati i po mogućnosti ponuditi rješenje kojim bi se tijek obrasca vratio na osnovni tijek.}\\
					

					\noindent \underbar{\textbf{UC$<$broj obrasca$>$ -$<$ime obrasca$>$}}
					\begin{packed_item}
	
						\item \textbf{Glavni sudionik: }$<$sudionik$>$
						\item  \textbf{Cilj:} $<$cilj$>$
						\item  \textbf{Sudionici:} $<$sudionici$>$
						\item  \textbf{Preduvjet:} $<$preduvjet$>$
						\item  \textbf{Opis osnovnog tijeka:}
						
						\item[] \begin{packed_enum}
	
							\item $<$opis korak jedan$>$
							\item $<$opis korak dva$>$
							\item $<$opis korak tri$>$
							\item $<$opis korak četiri$>$
							\item $<$opis korak pet$>$
						\end{packed_enum}
						
						\item  \textbf{Opis mogućih odstupanja:}
						
						\item[] \begin{packed_item}
	
							\item[2.a] $<$opis mogućeg scenarija odstupanja u koraku 2$>$
							\item[] \begin{packed_enum}
								
								\item $<$opis rješenja mogućeg scenarija korak 1$>$
								\item $<$opis rješenja mogućeg scenarija korak 2$>$
								
							\end{packed_enum}
							\item[2.b] $<$opis mogućeg scenarija odstupanja u koraku 2$>$
							\item[3.a] $<$opis mogućeg scenarija odstupanja  u koraku 3$>$
							
						\end{packed_item}
					\end{packed_item}
					
					\noindent \underbar{\textbf{UC1 - Registracija}}
					\begin{packed_item}
	
						\item \textbf{Glavni sudionik: }Korisnik
						\item  \textbf{Cilj:} Stvoriti korisnički račun
						\item  \textbf{Sudionici:} Baza podataka
						\item  \textbf{Preduvjet:} -
						\item  \textbf{Opis osnovnog tijeka:}
						
						\item[] \begin{packed_enum}
	
							\item Korisnik odabere opciju za registraciju
							\item Korisnik uspješno unese tražene podatke
							\item Korisnik dobiva obavijest o uspješnoj registraciji
							\item Korisnik biva preusmjeren na početnu stranicu
						\end{packed_enum}
						
						\item  \textbf{Opis mogućih odstupanja:}
						
						\item[] \begin{packed_item}
	
							\item[2.a] Korisnik odabere e-mail s kojim je već povezan korisnički račun u sustavu, ili unese tražene podatke u krivom formatu
							\item[] \begin{packed_enum}
								
								\item Korisnik dobiva obavijest o pogrešci
								\item Korisnik mijenja podatke i ponovno odabire opciju registracije, ili odustaje od registracije
								
							\end{packed_enum}
							
						\end{packed_item}
					\end{packed_item}
					
					\noindent \underbar{\textbf{UC2 - Prijava u sustav}}
					\begin{packed_item}
	
						\item \textbf{Glavni sudionik: }Korisnik
						\item  \textbf{Cilj:} Dobiti pristup sustavu
						\item  \textbf{Sudionici:} Baza podataka
						\item  \textbf{Preduvjet:} -
						\item  \textbf{Opis osnovnog tijeka:}
						
						\item[] \begin{packed_enum}
	
							\item Korisnik odabire opciju prijave u sustav
							\item Korisnik unosi e-mail adresu i lozinku
							\item Korisnik dobiva pristup sustavu
							\item Korisnik biva preusmjeren na početnu stranicu
						\end{packed_enum}
						
						\item  \textbf{Opis mogućih odstupanja:}
						
						\item[] \begin{packed_item}
	
							\item[3.a] Korisnik unosi e-mail s kojim nije povezan niti jedan korisnički račun, ili unosi krivu lozinku, ili unosi tražene podatke u krivom formatu
							\item[] \begin{packed_enum}
								
								\item Korisnik dobiva obavijest o pogrešci
								\item Korisnik mijenja podatke i ponovno odabire opciju prijave, ili odustaje od prijave
								
							\end{packed_enum}
							
						\end{packed_item}
					\end{packed_item}
					
					
					\noindent \underbar{\textbf{UC3 - Odjava iz sustava}}
					\begin{packed_item}
	
						\item \textbf{Glavni sudionik: }Korisnik
						\item  \textbf{Cilj:} Odjaviti se iz sustava
						\item  \textbf{Sudionici:} -
						\item  \textbf{Preduvjet:} Korisnik je prijavljen u sustav
						\item  \textbf{Opis osnovnog tijeka:}
						
						\item[] \begin{packed_enum}
	
							\item Korisnik odabire opciju odjave iz sustava
							\item Korisnik biva odjavljen iz sustava i preusmjeren na stranicu "Prijava"
						\end{packed_enum}
					\end{packed_item}					
					
					
					\noindent \underbar{\textbf{UC4 - Pregled osobnih podataka}}
					\begin{packed_item}
	
						\item \textbf{Glavni sudionik: }Korisnik
						\item  \textbf{Cilj:} Vidjeti osobne podatke
						\item  \textbf{Sudionici:} Baza podataka
						\item  \textbf{Preduvjet:} Korisnik je prijavljen u sustav
						\item  \textbf{Opis osnovnog tijeka:}
						
						\item[] \begin{packed_enum}
	
							\item Korisnik odabire opciju "Osobni podaci"
							\item Korisnik dobije prikaz osobnih podataka
						\end{packed_enum}
					\end{packed_item}
					
					\noindent \underbar{\textbf{UC5 - Promjena osobnih podataka}}
					\begin{packed_item}
	
						\item \textbf{Glavni sudionik: }Korisnik
						\item  \textbf{Cilj:} Promijeniti osobne podatke
						\item  \textbf{Sudionici:} Baza podataka
						\item  \textbf{Preduvjet:} Korisnik je prijavljen u sustav
						\item  \textbf{Opis osnovnog tijeka:}
						
						\item[] \begin{packed_enum}
	
							\item Korisnik odabire opciju promjene osobnih podataka
							\item Korisnik mijenja svoje osobne podatke
							\item Korisnik sprema promjene
							\item Baza podataka se ažurira
						\end{packed_enum}
						
						\item  \textbf{Opis mogućih odstupanja:}
						
						\item[] \begin{packed_item}
	
							\item[2.a] Korisnik mijenja svoje osobne podatke, ali ne spremi promjene
							\item[] \begin{packed_enum}
								
								\item Sustav upozori korisnika da nije spremio promjene
								\item Korisnik spremi promjene, ili odustane od njih
								
							\end{packed_enum}
							
							\item[2.b] Korisnik pokuša promijeniti e-mail adresu
							\item[] \begin{packed_enum}
								
								\item Sustav upozori korisnika da ne može promijeniti e-mail adresu
								\item Korisnik promijeni neke druge osobne podatke, ili odustane od promjena
								
							\end{packed_enum}
							
							\item[2.c] Korisnik pokuša promijeniti neki osobni podatak u nedozvoljeni format (npr. ime koje počinje malim slovom), ili pokuša promijeniti adresu na neku adresu koja ne postoji u sustavu
							\item[] \begin{packed_enum}
								
								\item Sustav upozori korisnika o pogrešci
								\item Korisnik pokuša promijeniti svoje osobne podatke u ispravan format, ili odustane od promjena
								
							\end{packed_enum}
							
						\end{packed_item}
					\end{packed_item}
					
					\noindent \underbar{\textbf{UC6 - Brisanje korisničkog računa}}
					\begin{packed_item}
	
						\item \textbf{Glavni sudionik: }Korisnik
						\item  \textbf{Cilj:} Obrisati korisnički račun
						\item  \textbf{Sudionici:} Baza podataka
						\item  \textbf{Preduvjet:} Korisnik je prijavljen u sustav
						\item  \textbf{Opis osnovnog tijeka:}
						
						\item[] \begin{packed_enum}
	
							\item Korisnik odabire opciju pregleda osobnih podataka
							\item Korisnik bira opciju brisanja korisničkog računa
							\item Sustav upozori korisnika da je brisanje računa trajna i nepovratna akcija
							\item Korisnik potvrđuje svoj odabir
							\item Korisnički račun se izbriše iz baze podataka
							\item Korisnik biva preusmjeren na stranicu za prijavu
							
						\end{packed_enum}
						
						\item  \textbf{Opis mogućih odstupanja:}
						
						\item[] \begin{packed_item}
	
							\item[4.a] Korisnik odustane od brisanja svog računa
							\item[] \begin{packed_enum}
								
								\item Korisnik biva preusmjeren na stranicu "Osobni podaci"
														
							\end{packed_enum}
							
						\end{packed_item}
					\end{packed_item}

					\noindent \underbar{\textbf{UC7 - Pregled događaja u cjelini "Događaji" }}
					\begin{packed_item}
	
						\item \textbf{Glavni sudionik: }Korisnik
						\item  \textbf{Cilj:} Prikaz potvrđenih događaja u korisnikovoj četvrti
						\item  \textbf{Sudionici:} Baza podataka
						\item  \textbf{Preduvjet:} Korisnik je prijavljen u sustav
						\item  \textbf{Opis osnovnog tijeka:}
						
						\item[] \begin{packed_enum}
	
							\item Korisnik odabere opciju "Događaji"
							\item Aplikacija prikazuje događaje za korisnikovu četvrt
						\end{packed_enum}
						
							
						\end{packed_item}
					\noindent \underbar{\textbf{UC8 - Stvaranje prijedloga događaja}}
					\begin{packed_item}
	
						\item \textbf{Glavni sudionik: }Korisnik
						\item  \textbf{Cilj:} Stvoriti prijedlog događaja
						\item  \textbf{Sudionici:} Baza podataka
						\item  \textbf{Preduvjet:} Korisnik je prijavljen u sustav
						\item  \textbf{Opis osnovnog tijeka:}
						
						\item[] \begin{packed_enum}
	
							\item Korisnik odabire opciju "Novi prijedlog događaja" 
							\item Korisnik ispunjava polja "naziv", "mjesto", "vrijeme", "trajanje" i "kratki opis"
							\item Korisnik stvara novi prijedlog događaja
							\item Baza podataka se ažurira
						\end{packed_enum}
						
						\item  \textbf{Opis mogućih odstupanja:}
						
						\item[] \begin{packed_item}
	
							\item[2.a] Korisnik nije ispunio sva polja prilikom stvaranja prijedloga događaja
							\item[] \begin{packed_enum}
								
								\item Sustav upozori korisnika da sva polja nisu ispunjena
								\item Korisnik ispuni preostala polja i objavi prijedlog događaja, ili odustane od prijedloga
								
							\end{packed_enum}
							
							
						\end{packed_item}
					\end{packed_item}						
					\noindent \underbar{\textbf{UC9 - Pregled prijedloga događaja}}
					\begin{packed_item}
	
						\item \textbf{Glavni sudionik: }Moderator
						\item  \textbf{Cilj:} Pregled prijedloga događaja u cjelini "Događaji"
						\item  \textbf{Sudionici:} Baza podataka
						\item  \textbf{Preduvjet:} Korisnik je prijavljen u sustav i ima ulogu "Moderator"
						\item  \textbf{Opis osnovnog tijeka:}
						
						\item[] \begin{packed_enum}
	
							\item Korisnik odabire opciju "Događaji" 
							\item Korisnik odabire opciju "Prijedlozi događaja" 
							\item Prikazuju se svi prijedlozi događaja za korisnikovu četvrt
							
						\end{packed_enum}
						\end{packed_item}
						
						\noindent \underbar{\textbf{UC10 - Uređivanje prijedloga događaja}}
					\begin{packed_item}
	
						\item \textbf{Glavni sudionik: }Moderator
						\item  \textbf{Cilj:} Uređivanje prijedloga događaja u cjelini "Događaji"
						\item  \textbf{Sudionici:} Baza podataka
						\item  \textbf{Preduvjet:} Korisnik je prijavljen u sustav i ima ulogu "Moderator"
						\item  \textbf{Opis osnovnog tijeka:}
						
						\item[] \begin{packed_enum}
	
							\item Korisnik odabire prijedlog događaja 
							\item Korisnik uređuje polja za odabrani prijedlog događaja
							\item Korisnik sprema promjene
							\item Baza podataka se ažurira
							
						\end{packed_enum}
						\item  \textbf{Opis mogućih odstupanja:}
						
						\item[] \begin{packed_item}
	
							\item[3.a] Moderator nije spremio promjene
							\item[] \begin{packed_enum}
								
								\item Sustav upozori moderatora da promjene nisu spremljene
								
							\end{packed_enum}
							
							
						\end{packed_item}
						\end{packed_item}
					\noindent \underbar{\textbf{UC11 - Objava prijedloga događaja}}
					\begin{packed_item}
	
						\item \textbf{Glavni sudionik: }Moderator
						\item  \textbf{Cilj:} Objava predloženog događaja u cjelini "Događaji"
						\item  \textbf{Sudionici:} Baza podataka
						\item  \textbf{Preduvjet:} Korisnik je prijavljen u sustav i ima ulogu "Moderator"
						\item  \textbf{Opis osnovnog tijeka:}
						
						\item[] \begin{packed_enum}
	
							\item Korisnik odabire prijedlog događaja 
							\item Korisnik objavljuje događaj u cjelini "Događanja"
							\item Baza podataka se ažurira
							
						\end{packed_enum}
						\end{packed_item}
					\noindent \underbar{\textbf{UC12 - Brisanje prijedloga događaja}}
					\begin{packed_item}
	
						\item \textbf{Glavni sudionik: }Moderator
						\item  \textbf{Cilj:} Brisanje prijedloga događaja u cjelini "Događaji"
						\item  \textbf{Sudionici:} Baza podataka
						\item  \textbf{Preduvjet:} Korisnik je prijavljen u sustav i ima ulogu "Moderator"
						\item  \textbf{Opis osnovnog tijeka:}
						
						\item[] \begin{packed_enum}
	
							\item Korisnik odabire prijedlog događaja 
							\item Korisnik briše prijedlog događaja 
							\item Baza podataka se ažurira
							
						\end{packed_enum}
						
					\end{packed_item}
					\noindent \underbar{\textbf{UC13 - Pregled tema na forumu}}
					\begin{packed_item}
	
						\item \textbf{Glavni sudionik: }Korisnik
						\item  \textbf{Cilj:} Prikaz tema na forumu
						\item  \textbf{Sudionici:} Baza podataka
						\item  \textbf{Preduvjet:} Korisnik je prijavljen u sustav
						\item  \textbf{Opis osnovnog tijeka:}
						
						\item[] \begin{packed_enum}
	
							\item Korisnik odabire cjelinu „Forum“
							\item Prikazuju se teme na forumu poredane po vremenu zadnjeg odgovora
							
						\end{packed_enum}
						
						
					\end{packed_item}						
					\noindent \underbar{\textbf{UC14 - Prikaz teme na forumu}}
					\begin{packed_item}
	
						\item \textbf{Glavni sudionik: }Korisnik
						\item  \textbf{Cilj:} Otvoriti temu na forumu i vidjeti sve objave u njoj
						\item  \textbf{Sudionici:} Baza podataka
						\item  \textbf{Preduvjet:} Korisnik je prijavljen u sustav
						\item  \textbf{Opis osnovnog tijeka:}
						
						\item[] \begin{packed_enum}
	
							\item Korisnik u cjelini "Forum" odabire željenu temu
							\item Prikazuje se odabrana tema
							
						\end{packed_enum}
						
						
					\end{packed_item}
					\noindent \underbar{\textbf{UC15 - Odabir objave za odgovor }}
					\begin{packed_item}
	
						\item \textbf{Glavni sudionik: }Korisnik
						\item  \textbf{Cilj:} Odabir objave na koju korisnik želi odgovoriti
						\item  \textbf{Sudionici:} Baza podataka
						\item  \textbf{Preduvjet:} Korisnik je prijavljen u sustav
						\item  \textbf{Opis osnovnog tijeka:}
						
						\item[] \begin{packed_enum}
	
							\item Korisnik unutar teme na forumu na odabranoj objavi odabire opciju "Odgovori"
							\item Otvara se polje za odgovor koje odgovara na željenu objavu
							
							
						\end{packed_enum}
						
						
					\end{packed_item}
					\noindent \underbar{\textbf{UC16 - Uređivanje odgovora }}
					\begin{packed_item}
	
						\item \textbf{Glavni sudionik: }Korisnik
						\item  \textbf{Cilj:} Uređivanje odgovora u temi na forumu 
						\item  \textbf{Sudionici:} Baza podataka
						\item  \textbf{Preduvjet:} Korisnik je prijavljen u sustav
						\item  \textbf{Opis osnovnog tijeka:}
						
						\item[] \begin{packed_enum}
	
							\item Korisnik u polju za odgovore uređuje tekst
						\end{packed_enum}
							
						\item  \textbf{Opis mogućih odstupanja:}
						
						\item[] \begin{packed_item}
						\item[2.a] Korisnik napušta stranicu iako napisani odgovor nije objavio
							\item[] \begin{packed_enum}
								
								\item Sustav upozorava korisnika da uređeni tekst neće biti spremljen ako napusti stranicu
								
							\end{packed_enum}
						\end{packed_item}
						
					\end{packed_item}	
					\noindent \underbar{\textbf{UC17 - Objava odgovora }}
					\begin{packed_item}
	
						\item \textbf{Glavni sudionik: }Korisnik
						\item  \textbf{Cilj:} Objava skice napisane u polju za uređivanje teksta
						\item  \textbf{Sudionici:} Baza podataka
						\item  \textbf{Preduvjet:} Korisnik je prijavljen u sustav
						\item  \textbf{Opis osnovnog tijeka:}
						
						\item[] \begin{packed_enum}
	
							\item Korisnik odabire opciju "Objavi" 
							\item Baza podataka se ažurira
							\item Vidljivi odgovori unutar teme se ažuriraju
							
							
						\end{packed_enum}
						\item  \textbf{Opis mogućih odstupanja:}
						
						\item[] \begin{packed_item}
						\item[2.a] Korisnik odabire opciju objava iako je polje za odgovor prazno
							\item[] \begin{packed_enum}
								
								\item Sustav upozorava korisnika da je polje za odgovor prazno
								\item Korisnik ispunjava polje za odgovor, ili odustane od odgovora
								
							\end{packed_enum}
						\end{packed_item}
						
						
					\end{packed_item}					
										
					\noindent \underbar{\textbf{UC18 - Brisanje odgovora }}
					\begin{packed_item}
	
						\item \textbf{Glavni sudionik: }Korisnik
						\item  \textbf{Cilj:} Brisanje odgovora u temi
						\item  \textbf{Sudionici:} Baza podataka
						\item  \textbf{Preduvjet:} Korisnik je prijavljen u sustav, korisnik je autor odgovora ili korisnik ima ulogu "Moderator"
						\item  \textbf{Opis osnovnog tijeka:}
						
						\item[] \begin{packed_enum}
	
							\item Korisnik pronalazi odgovor koji želi obrisati
							\item Odabire opciju "Izbriši"
							\item Sustav upozorava korisnika da je brisanje odgovora trajna i nepovratna akcija
							\item Korisnik potvrđuje svoj odabir
							\item Baza podataka se ažurira
							\item Vidljivi odgovori unutar teme se ažuriraju
							
							
						\end{packed_enum}
						
						\item  \textbf{Opis mogućih odstupanja:}
						
						\item[] \begin{packed_item}
						\item[4.a] Korisnik odustane od brisanja odgovora
							\item[] \begin{packed_enum}
								
								\item Prozor za brisanje odgovora se ugasi
								
							\end{packed_enum}
						\end{packed_item}
						
						
					\end{packed_item}					
					\noindent \underbar{\textbf{UC19 - Pregled cjeline "Vijeće četvrti" }}
					\begin{packed_item}
	
						\item \textbf{Glavni sudionik: }Korisnik
						\item  \textbf{Cilj:} Pregled izvješća i budućih tema
						\item  \textbf{Sudionici:} Baza podataka
						\item  \textbf{Preduvjet:} Korisnik je prijavljen u sustav
						\item  \textbf{Opis osnovnog tijeka:}
						
						\item[] \begin{packed_enum}
	
							\item Korisnik odabire opciju "Vijeće četvrti" 
							\item Prikazuju se objave Vijeća četvrti, poredane po vremenu objave
							
							
						\end{packed_enum}
						
						
					\end{packed_item}					
					\noindent \underbar{\textbf{UC20 - Stvaranje novog izvješća }}
					\begin{packed_item}
	
						\item \textbf{Glavni sudionik: }Korisnik
						\item  \textbf{Cilj:} Otvaranje polja za pisanje izvješća
						\item  \textbf{Sudionici:} Baza podataka
						\item  \textbf{Preduvjet:} Korisnik je prijavljen u sustav i ima ulogu "Vijećnik"
						\item  \textbf{Opis osnovnog tijeka:}
						
						\item[] \begin{packed_enum}
	
							\item Korisnik odabire opciju "Stvaranje novog izvješća" 
							\item Otvara se polje za pisanje izvješća
							
							
						\end{packed_enum}
						
						
					\end{packed_item}					
					\noindent \underbar{\textbf{UC21 - Uređivanje izvješća }}
					\begin{packed_item}
	
						\item \textbf{Glavni sudionik: }Vijećnik
						\item  \textbf{Cilj:} Urediti kratko izvješće
						\item  \textbf{Sudionici:} Baza podataka
						\item  \textbf{Preduvjet:} Korisnik je prijavljen u sustav i ima ulogu "Vijećnik"
						\item  \textbf{Opis osnovnog tijeka:}
						
						\item[] \begin{packed_enum}
	
							\item Korisnik uređuje izvješće u tekstualnom polju 
							\item Korisnik sprema promjene na izvješću
							\item Baza podataka se ažurira
							
							
							
						\end{packed_enum}
						\item  \textbf{Opis mogućih odstupanja:}
						
						\item[] \begin{packed_item}
						\item[2.a] Korisnik zatvara stranicu bez spremanja promjena
							\item[] \begin{packed_enum}
								
								\item Sustav upozorava korisnika želi li spremiti napisani tekst
								
							\end{packed_enum}
						\end{packed_item}
						
						
						
					\end{packed_item}					
					

					\noindent \underbar{\textbf{UC22 - Objava izvješća }}
					\begin{packed_item}
	
						\item \textbf{Glavni sudionik: }Vijećnik
						\item  \textbf{Cilj:} Objava kratkog izvješća
						\item  \textbf{Sudionici:} Baza podataka
						\item  \textbf{Preduvjet:} Korisnik je prijavljen u sustav i ima ulogu "Vijećnik"
						\item  \textbf{Opis osnovnog tijeka:}
						
						\item[] \begin{packed_enum}
	
							\item Korisnik odabire opciju "Objavi" 
							\item Izvješće s "Vijeća četvrti" se objavljuje
							\item Baza podataka se ažurira
							\item Prikaz izvješća na "Vijeću četvrti" se ažurira	
							
							
						\end{packed_enum}
						\item  \textbf{Opis mogućih odstupanja:}
						
						\item[] \begin{packed_item}
						\item[2.a] Korisnik odabire opciju "objavi" iako je tekstualno polje prazno
							\item[] \begin{packed_enum}
								
								\item Sustav upozorava korisnika da je polje za tekst prazno
								
							\end{packed_enum}
						\end{packed_item}
						
						
						
					\end{packed_item}
					\noindent \underbar{\textbf{UC23 - Otvaranje teme na forumu vezane za izvješće s "Vijeća četvrti"}}
					\begin{packed_item}
	
						\item \textbf{Glavni sudionik: }Korisnik
						\item  \textbf{Cilj:} Na forumu stvoriti temu vezanu uz izvješće na "Vijeću četvrti"
						\item  \textbf{Sudionici:} Baza podataka
						\item  \textbf{Preduvjet:} Korisnik je prijavljen u sustav, za izvješće za koje korisnik želi otvoriti temu na forumu nije već otvorena tema
						\item  \textbf{Opis osnovnog tijeka:}
						
						\item[] \begin{packed_enum}
	
							\item Korisnik odabire opciju "Otvori temu na forumu" kod izvješća u cjelini "Vijeće četvrti
							\item Otvara se nova tema na cjelini "Forum"
							\item Baza podataka se ažurira
							
						\end{packed_enum}				
						
					\end{packed_item}
					
					\noindent \underbar{\textbf{UC24 - Podnošenje zahtjeva za dodjelu uloge}}
					\begin{packed_item}
	
						\item \textbf{Glavni sudionik: }Korisnik
						\item  \textbf{Cilj:} Priložiti zahtjev za promjenu uloge
						\item  \textbf{Sudionici:} Baza podataka
						\item  \textbf{Preduvjet:} Korisnik nema ulogu za koju šalje zahtjev
						\item  \textbf{Opis osnovnog tijeka:}
						
						\item[] \begin{packed_enum}
	
							\item Korisnik odabire opciju "Osobni podaci"
							\item Korisnik odabire opciju "Zahtjev za ulogom"
							\item Korisnik odabire ulogu za koju šalje zahtjev
							\item Korisnik pošalje zahtjev
							\item Baza podataka se ažurira
						\end{packed_enum}
						
						\item  \textbf{Opis mogućih odstupanja:}
						
						\item[] \begin{packed_item}
						
							\item[3.a] Korisnik pokuša poslati zahtjev bez da je odabrao ulogu
							\item[] \begin{packed_enum}
								
								\item Sustav upozori korisnika da nije odabrao ulogu
								\item Korisnik odabere ulogu, ili odustane od zahtjeva
								
							\end{packed_enum}
	
							\item[4.a] Korisnik ugasi prozor za zahtjev za ulogu bez da je poslao zahtjev
							\item[] \begin{packed_enum}
								
								\item Sustav upozori korisnika da nije poslao zahtjev
								\item Korisnik pošalje zahtjev, ili odustane od njega
								
							\end{packed_enum}
							
						\end{packed_item}
					\end{packed_item}
						
				\subsubsection{Dijagrami obrazaca uporabe}
					
					\textit{Prikazati odnos aktora i obrazaca uporabe odgovarajućim UML dijagramom. Nije nužno nacrtati sve na jednom dijagramu. Modelirati po razinama apstrakcije i skupovima srodnih funkcionalnosti.}
				\eject		
				
			\subsection{Sekvencijski dijagrami}
				
				\textbf{\textit{dio 1. revizije}}\\
				
				\textit{Nacrtati sekvencijske dijagrame koji modeliraju najvažnije dijelove sustava (max. 4 dijagrama). Ukoliko postoji nedoumica oko odabira, razjasniti s asistentom. Uz svaki dijagram napisati detaljni opis dijagrama.}
				\eject
	
		\section{Ostali zahtjevi}
		
			\textbf{\textit{dio 1. revizije}}\\
		 
			 \textit{Nefunkcionalni zahtjevi i zahtjevi domene primjene dopunjuju funkcionalne zahtjeve. Oni opisuju \textbf{kako se sustav treba ponašati} i koja \textbf{ograničenja} treba poštivati (performanse, korisničko iskustvo, pouzdanost, standardi kvalitete, sigurnost...). Primjeri takvih zahtjeva u Vašem projektu mogu biti: podržani jezici korisničkog sučelja, vrijeme odziva, najveći mogući podržani broj korisnika, podržane web/mobilne platforme, razina zaštite (protokoli komunikacije, kriptiranje...)... Svaki takav zahtjev potrebno je navesti u jednoj ili dvije rečenice.}
			 
			 
			 
	