\chapter{Zaključak i budući rad}
		
		Zadatak naše grupe bio je razvoj web aplikacije pod nazivom "Moj kvart". Ideja aplikacije je da bude svojevrsna društvena mreža preko koje stanovnici istih kvartova mogu komunicirati, komentirati neke teme na forumu, predlagati grupne događaje i slično. Nakon 12 tjedana timskog rada, ostvarili smo zadani cilj i projekt je završen. Projekt je imao tri faze.
		
		Prva faza je uključivala okupljanje tima, razgovor o generalnim idejama za aplikaciju, izražavanje pojedinačnih interesa i želja za radom u prvom ciklusu predaje, te konačno podjela zadataka za prvi ciklus. Formirala su se dva podtima. U prvom podtimu su članovi radili na dokumentiranju zahtjeva, obrazaca uporabe, UML dijagrama te bazi podataka. U drugom podtimu su članovi istraživali, proučavali i eskperimentirali s tehnologijama u kojima će kasnije biti implementirana aplikacija. Prva je faza projekta trajala do kolokviranja prvog ciklusa projekta.
		
		U drugoj je fazi naglasak bio na implementaciji aplikacije, i ovdje nije bilo podtimova nego su svi članovi zajedno radili na programskom ostvarenju aplikacije. U ovoj je fazi implementirana većina aplikacije, a trajala je do demonstracije alfa inačice aplikacije.
		
		Treća i konačna faza je uključivala izradu raznih UML dijagrama, ispitivanje sustava, pronalazak i ispravak grešaka, rad na izgledu aplikacije i implementacija preostalih funkcionalnosti. U ovoj fazi je postojalo dosta manjih zadataka koje je trebalo napraviti, pa su članovi tima uglavnom samostalno preuzimali te zadatke i obavljali ih. Ova je faza trajala do završetka projekta, prije konačne predaje i kolokviranja drugog ciklusa.
		
		Izgrađenu aplikaciju je moguće proširiti na mnogo načina. Jedna od ideja je u Forum dodati više funkcionalnosti, tako da korisnici mogu pregledati druge korisnike i sve njihove objave, te da dobivaju obavijesti kada im netko odgovori na objavu. Drugo proširenje koje bi doprinijelo kvaliteti aplikacije je mogućnost da korisnici šalju izravno poruke drugim korisnicima.
		
		Zbog problema s Google Calendar API-jem, nije implementirana sinkronizacija događaja unutar aplikacije s Google Kalendarom.
		
		Sudjelovanje u ovom projektu je bilo vrijedno iskustvo za sve članove tima. Svima nama je ovo bio prvi ozbiljniji grupni projekt, i snašli smo se jako dobro. Konflikata u timu gotovo da nije bilo, a suradnja i komunikacija su bili na iznimno zadovoljavajućoj razini. Većini nas je ovaj projekt bio prvi ozbiljniji dodir s tehnologijama poput Gita i Latex. Naučili smo koristiti neke moderne radne okvire pri izradi web aplikacija. Iznimno smo zadovoljni postignutim rezultatima i timskim radom koji je do tih rezultata doveo.
		
		\eject 